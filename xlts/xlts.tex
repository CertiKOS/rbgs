\documentclass[mathserif]{beamer}
\usepackage{bbm}
\usepackage{cmll}
\usepackage{tikz-cd}
\newcommand{\que}{\circ}
\newcommand{\ans}{\bullet}

\title{Extended transition system models for CompCertO+}

\begin{document}

\maketitle

\begin{frame}{Problem statement}
  CompCertO's transition systems
  have various limitations:
  \begin{itemize}
    \item ad-hoc definition (global environments, silent divergence, etc);
    \item limited to sequential computation;
    \item further limited to signatures of the form $\{ m: A^\ans \mid m \in A^\que \}$;
    \item no persistent state w/o CompCertOE constructions;
    \item no reentrency even with CompCertOE;
  \end{itemize}

  \vspace{1em}
  We want to overcome them
  and explore possible generalizations.
\end{frame}

\begin{frame}{Starting point}
  In CompCertO,
  $L = \langle S, {\rightarrow}, I, X, Y, F \rangle : A \twoheadrightarrow B$
  consists of:
  \begin{align*}
    I &\:\subseteq\: B^\que \times S
    &
    F &\:\subseteq\: S \times B^\ans
    \\
    X &\:\subseteq\: S \times A^\que
    &
    Y^{s} &\:\subseteq\: A^\ans \times S
  \end{align*}
  for the \emph{language interfaces} $A = \langle A^\que, A^\ans \rangle$
  and $B = \langle B^\que, B^\ans \rangle$.

  \pause
  \vfill
  Equivalently, this can be described as a coalgebra in $\mathbf{Rel}$,
  \begin{align*}
    \langle {\rightarrow}, X, Y, F \rangle
      : {} &S \rightarrow S \oplus (A^\que \with S^{A^\ans}) \oplus B^\ans \in \mathbf{Rel} \\
      \subseteq {} & S \times
      (S + A^\que + {A^\ans} \times S + B^\ans)
    \,,
  \end{align*}
  with a relation $I : B^\que \rightarrow S \in \mathbf{Rel}$
  identifying starting states.

  \pause
  \vfill
  The functor $FX := X \oplus (A^\que \with X^{A^\ans}) \oplus B^\ans$
  is also described by:
  \[
    \Sigma :=
    \{ \tau : \mathbbm{1} \} \:\cup\:
    \{ m : A^\ans \mid m \in A^\que \} \:\cup\:
    \{ r : \varnothing \mid r \in B^\ans \}
  \]
\end{frame}

\begin{frame}{Methodology}
  I am in the process of exploring different approaches \\
  to generalize CompCertO's transition systems.

  \pause
  \vfill
  My template for Coq experiments is to try to:
  \begin{itemize}
    \item define $\mathsf{id}, {\circ}$
    \item define simulations
    \item prove the associated propertites (left unit, right unit, assoc)
  \end{itemize}
  Then maybe more CompCertO-type stuff, concurrent games.

  \vfill
  I will document the ideas and results in these slides.
\end{frame}

\section{Coalgebraic approach}

\frame\sectionpage

\begin{frame}{The need for multiple sorts}
  The coalgebraic reading of transition systems offers hope of \\
  a less ad-hoc model and opportunities for generalization:
  \begin{itemize}
    \item from a fixed computation shape to more general forms of $F$;
    \item use coalgebra to make things rigorous as they get complicated;
    \item can draw on lots of existing research.
  \end{itemize}

  \pause
  \vfill
  However, the simple version has a fundamental limitation:
  \begin{itemize}
    \item in algebraic terms, the signature is \emph{single-sorted};
    \item in operational terms, there is only one \emph{type of state}.
  \end{itemize}

  \pause
  \vfill
  Unfortunately, this prevents a natural treatment of things such as:
  \begin{itemize}
    \item effect signatures,
    \item built-in persistent state,
    \item reentrency and concurrency.
  \end{itemize}
\end{frame}

\begin{frame}[fragile]{Multi-sorted transition systems}
  We want states to be indexed by the current ``phase'' we are in:
  \begin{itemize}
    \item a persistent state in $S_*$ waits for an incoming question;
    \item a question $q \in B^\que$ switches to a state in $S_q$;
    \item a state in $S_{qm}$ when we are waiting for $m \in A^\que$ to return.
  \end{itemize}
  \[
    \begin{tikzcd}[column sep=large]
      S_* \ar[r, "I/q", shift left] &
      S_q \ar[r, "X_q/m", shift left, pos=0.55]
          \ar[l, "F_q/r", shift left]
          \ar[loop above, "\tau", <-] &
      S_{qm} \ar[l, "Y_{qm}/n", shift left, pos=0.45]
    \end{tikzcd}
  \]

  \pause
  \vfill
  In an coalgebraic context this corresponds to having multiple \emph{sorts}:
  \begin{gather*}
    T := \{ *, q, qm \mid q \in B^\que, m \in A^\que \} \qquad
    F : \mathbf{Rel}^T \rightarrow \mathbf{Rel}^T \\[1ex]
    F_*(\mathbf{S}) := \prod_{q \in B^\que} S_q \qquad \qquad
    F_{qm}(\mathbf{S}) := \prod_{n \in A^\ans_m} S_q \\
    F_q(\mathbf{S}) := S_q \:+\: \sum_{r \in B^\ans_q} S_* \:+\: \sum_{m \in A^\que} S_{qm} \\
  \end{gather*}
  \vspace{-2em}
\end{frame}

\begin{frame}[fragile]{Evaluation}
  Implementing in Coq,
  there are positive aspects:
  \begin{itemize}
    \item a good match for dependent types;
    \item in a $\mathbf{Set}$ version things are very pleasant.
  \end{itemize}
  However, some issues show up:
  \begin{itemize}
    \item won't be expressive enough for concurrency;
    \item the $\mathbf{Rel}$ version is not quite as enjoyable.
  \end{itemize}

  \pause
  \vfill
  Simulations can be defined using the morphism part of $F$ \\
  but the functoriality properties are a lot to manage.
  \[
    \begin{tikzcd}[sep=0]
      A \ar[rr, "\alpha"] &&
      F A \\
      & \subseteq & \\
      B \ar[rr, "\beta"'] \ar[uu, "R"] &&
      F B \ar[uu, "F R"']
    \end{tikzcd}
    \qquad
    \exists R \mathbin.
    \alpha \circ R  \subseteq  F R \circ \beta
  \]
\end{frame}

\section{Multi-sorted signatures}

\frame\sectionpage


\end{document}

